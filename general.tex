\RequirePackage[l2tabu, orthodox]{nag}
\RequirePackage{silence}
\WarningFilter{nag}{There is no environment ``centering'' }%nag complains because beamer titlepage uses a centering environment
\WarningFilter{nag}{1 complaints in total}
\WarningFilter{ifpdf}{Someone has redefined \pdfoutput}
\WarningFilter{fmtcount}{\ordinal already defined use \FCordinal instead}
\documentclass[french,english]{beamer}
\input{preamble/packages}
\input{preamble/math_basics}
\input{preamble/math_mine}
\input{preamble/redac}
\input{preamble/draw}
\input{preamble/acronyms}

\title{Arguing about voting rules}
\subject{Social Choice}
\keywords{axioms, Borda, Condorcet, explanation, argumentation}
\author{Olivier Cailloux}
\institute[LAMSADE]{LAMSADE, Université Paris-Dauphine \\ Joint work with Ulle Endriss, University of Amsterdam}
\date{\formatdate{16}{2}{2017}}
\begin{document}
\begin{frame}[plain]
	\tikz[remember picture,overlay]{
		\path (current page.south west) node[anchor=south west, inner sep=0] {
			\includegraphics[height=1cm]{LAMSADE95.jpg}
		};
		\path (current page.south) ++ (0, 1mm) node[anchor=south, inner sep=0] {
			\includegraphics[height=9mm]{Dauphine.jpg}
		};
		\path (current page.south east) node[anchor=south east, inner sep=0] {
			\includegraphics[height=1cm]{PSL.png}
		};
	}
	\titlepage
\end{frame}
\addtocounter{framenumber}{-1}

\begin{frame}
	\frametitle{Outline}
	\tableofcontents[hideallsubsections, sectionstyle=shaded/show]
\end{frame}

\AtBeginSection{
	\begin{frame}
		\frametitle{Outline}
		\tableofcontents[currentsection, hideallsubsections]
	\end{frame}
}

\section{Introduction to Social Choice}
\subsection{Definition}
\begin{frame}
	\frametitle{\subsecname}
	\begin{definition}[Social Choice]
		Designing and analyzing methods for collective decision making
	\end{definition}
	\begin{itemize}
		\item Focus on: mathematical analysis {\tiny usually non empirical}
		\item (Usually) no focus on: collective social behavior
	\end{itemize}
\end{frame}

\subsection{The Condorcet paradox}
\begin{frame}
	\frametitle{\subsecname}
	\begin{itemize}
		\item Why should we analyze decision schemes mathematically?
		\item Here is a preliminary argument
		\item 3 alternatives : $a, b, c$
		\item 3 voters : $1, 2, 3$
		\item Compare alternatives pairwise
		\item $a$ VS $b$: 2 votes for $a$
		\item $b$ VS $c$: 2 votes for $b$
		\item Conclusion? \pause $a \succ c$? Not so simple!
	\end{itemize}
	\begin{equation}
		\begin{array}{rrr}
			a	&b	&c\\
			b	&c	&a\\
			c	&a	&b\\
		\end{array}
	\end{equation}
\end{frame}

\subsection{Two key moments}
\begin{frame}
	\frametitle{\subsecname: 1. Borda, Condorcet}
	\begin{itemize}
		\item Nicolas de Condorcet [1743 – 1794]
		\item Jean-Charles de Borda [1733 – 1799]
		\item Start of \emph{mathematical, systematic} analysis of voting schemes
		\item Rich period of constitution writing
		\item Condorcet initially involved in the writing of the French constitution
		\item Borda’s rank-order method was first proposed orally at the French Academy of Science in 1770 \citep{suzumura_introduction_2002}
	\end{itemize}
\end{frame}

\begin{frame}
	\frametitle{\subsecname: 2. Arrow}
	\begin{itemize}
		\item up to Arrow: mostly analysis of individual voting schemes
		\item Kenneth Arrow [1921 – ]
		\item Analysis of \emph{classes} of voting schemes
		\item According to properties
		\item[⇒] {\tiny (for example)} Nobel price in 1998 to Amartya Sen (welfare economics)
	\end{itemize}
\end{frame}

\subsection{This talk VS reality}
\begin{frame}
	\frametitle{\subsecname}
	\begin{itemize}
		\item Elections happen IRL (? \pause in real life) \pause
		\item Quite far from the setting examined here
		\item Preferences expressed in various ways
		\item Procedures may not fit this framework
		\item Social Choice in general relevant to this part of reality
		\item In this talk: \emph{limited} connexion
	\end{itemize}
\end{frame}

\section{Voting rules}
\subsection{Definition}
\begin{frame}
	\frametitle{\subsecname}
	\begin{itemize}
		\item Voting rule: a systematic way of aggregating different opinions and decide
		\item Multiple reasonable ways of doing this
		\item Different voting rules have different interesting properties
		\item None satisfy all desirable properties
	\end{itemize}
\end{frame}

\begin{frame}[fragile]
	\frametitle{Voting rule}
	
	\begin{description}[Alternatives]
		\item[Alternatives] $\allalts = \set{a, b, c, d, \ldots}$
		\item[Voters] $\allvoters = \set{1, 2, \ldots}$
		\item[Profile] function $\prof$ from $\allvoters$ to linear orders on $\allalts$.
		\item[Voting rule] function $f$ mapping each $\prof$ to winners $\emptyset \subset \alts \subseteq \allalts$.
	\end{description}
	\vfill
	\begin{center}
		\begin{tikzpicture}
			\path node[profile matrix] (profile) {
				\prof(1)&
				\prof(2)
				\\
				| (profile11) | a&
				b
				\\
				b&
				a
				\\
				c&
				| (profile32) | c
				\\
			};
			\path ($(profile.south west)!.5!(profile.south east)$) ++ (0, -5mm) node {$\prof$};
			
			\path node[draw, rectangle, fit=(profile11) (profile32), outer xsep=2mm, outer ysep=1mm] (justprofile) {};
			\path (justprofile.east) ++ (2.5cm, 0) node[inner sep=0] (winners) {\mbox{} $\alts = \Set{a, b}$};
			\path[draw, ->] (justprofile.east) to[bend left=35] node[anchor=south] {$f$} (winners.west);

			\path[draw, decorate, decoration={brace, mirror}] (justprofile.south west) -- (justprofile.south east);
		\end{tikzpicture}
	\end{center}
\end{frame}

\subsection{Example of a profile}
\begin{frame}[fragile]
	\frametitle{\subsecname}
	\begin{equation}
		\begin{array}{lrrrrrr}
			&\multicolumn{6}{c}{\text{nb voters}}\\
		\cmidrule{2-7}
				&33	&16	&3	&8	&18	&22	\\
		\midrule
			1	&a	&b	&c	&c	&d	&e	\\
			2	&b	&d	&d	&e	&e	&c	\\
			3	&c	&c	&b	&b	&c	&b	\\
			4	&d	&e	&a	&d	&b	&d	\\
			5	&e	&a	&e	&a	&a	&a	\\
		\end{array}
	\end{equation}
	Who wins?\pause
	\begin{itemize}
		\item Most top-1: $a$
		\item $c$ is in the top 3 for everybody
		\item delete worst first, lowest nb of pref: $c$, $b$, $e$, $a$ ⇒ $d$
		\item delete worst first, from bottom: $a$, $e$, $d$, $b$ ⇒ $c$
		\item Borda: $b$
%		\item Condorcet: $c$
	\end{itemize}
\end{frame}

\subsection{Two voting rules}
\begin{frame}
	\frametitle{Borda}
	
Given a profile $\prof$:
	\begin{itemize}
		\item score of $a \in \allalts$: number of alternatives it beats
		\item the highest scores win
	\end{itemize}
	
	\begin{equation}
		\prof =
		\begin{array}{rrrrr}
			a	&	a	&	a	&	b	&	b\\
			b	&	b	&	b	&	c	&	c\\
			c	&	c	&	c	&	a	&	a
		\end{array}
	\end{equation}
	\begin{itemize}
		\item score $a$ is\dots? \pause $2 + 2 + 2 = 6$
		\item score $b$ is $1 + 1 + 1 + 2 + 2 = 7$
		\item score $c$ is $1 + 1 = 2$
	\end{itemize}
	Winner: $b$.
\end{frame}

\begin{frame}
	\frametitle{Condorcet’s principle}
	\begin{block}{Condorcet’s principle}
		We ought to take the Condorcet winner as sole winner if it exists.
		\begin{itemize}
			\item $a$ \emph{beats} $b$ iff more than half the voters prefer $a$ to $b$.
			\item $a$ is a \emph{Condorcet winner} iff $a$ beats every other alternatives.
		\end{itemize}
	\end{block}
	\vfill
	\begin{equation}
		\prof =
		\begin{array}{rrrrr}
			a	&	a	&	a	&	b	&	b\\
			b	&	b	&	b	&	c	&	c\\
			c	&	c	&	c	&	a	&	a
		\end{array}
	\end{equation}
	 Who wins? \pause $a$
\end{frame}

\begin{frame}
	\frametitle{Condorcet’s principle and a voting rule}
	\begin{itemize}
		\item Condorcet’s principle does not define a voting rule. Why? \pause
		\item No winner is defined when no Condorcet winner
	\end{itemize}
	\begin{equation}
		\prof =
		\begin{array}{rrr}
			b	&	c	&	d\\
			c	&	d	&	b\\
			a	&	b	&	a\\
			d	&	a	&	c
		\end{array}
	\end{equation}
	\pause
	$a$ loses against $b$; $b$ against $d$; $c$ against $b$; $d$ against $c$
	\begin{itemize}
		\item Dodgson’s method (1876): candidates “closest” to being Condorcet winners {\tiny (in nb of swaps)}
	\end{itemize}
\end{frame}

\section{Analyzing voting rules}
\subsection[Problem]{Problems do arise}
\begin{frame}
	\frametitle{\subsecname}
	\begin{itemize}
		\item Example of a problem: Borda does not satisfy Condorcet’s principle
		\item Even when knowing the precise rule, we may overlook some problems of it
		\item We would like some way of finding out systematically whether a rule is “well-behaved”
	\end{itemize}
\end{frame}

\subsection{Axiomatics}
\begin{frame}
	\frametitle{\subsecname}
	\begin{quote}
		Rather than dream up a multitude of arbitration schemes and determine whether or not each withstands the best of plausibility in a host of special cases, let us invert the procedure. Let us examine our subjective intuition of fairness and formulate this as a set of precise desiderata that any acceptable arbitration scheme must fulfil. Once these desiderata are formalized as axioms, then the problem is reduced to a mathematical investigation of the existence of and characterization of arbitration schemes which satisfy the axioms.
	\end{quote}
	\citet[p. 121]{luce_games_1957}\par
	Some example of axioms? \pause Condorcet’s principle!
\end{frame}

\begin{frame}
	\frametitle{What’s an axiom?}
	\begin{itemize}
		\item An axiom (for us) is a principle
		\item Expressed formally
		\item That dictates some behavior of a voting rule
		\item In some conditions
		\item Usually seen as something to be satisfied
		\item Ideally, some union of some such axioms define exactly one rule
		\item Some axioms can be shown to be incompatible
	\end{itemize}
	\begin{example}[Condorcet’s principle]
		We ought to take the Condorcet winner as sole winner if it exists.
	\end{example}
\end{frame}

\subsection{Unanimity}
\begin{frame}
	\frametitle{\subsecname}
	\begin{definition}[Unanimity]
		We ought to select as winner someone who has no unanimously preferred alternative
	\end{definition}
	\begin{equation}
		\prof =
		\begin{array}{rrr}
			a	&	a	&	b\\
			b	&	b	&	c\\
			c	&	c	&	a
		\end{array}
	\end{equation}
	Constraint? \pause Do not take $c$ as $b$ is unanimously preferred to it.	\pause
	\begin{equation}
		\prof =
		\begin{array}{rrr}
			a	&	a	&	b\\
			b	&	c	&	c\\
			c	&	b	&	a
		\end{array}
	\end{equation}
	Constraint? \pause No constraint.
\end{frame}

\subsection[IIF]{Independence of Infeasible Alternatives}
\begin{frame}[fragile]
	\frametitle{Social choice correspondence}
%	\begin{description}[Social choice correspondence]
%		\item[Alternatives] $\allalts = \set{a, b, c, d, \ldots}$
%		\item[Feasible alternatives] $\emptyset \subset F \subseteq \allalts$
%		\item[Voters] $\allvoters = \set{1, 2, \ldots}$
%		\item[Profile] function $\prof$ from $\allvoters$ to orders on $\allalts$
%		\item[Social choice correspondence] function $f$ mapping each $\prof$ \emph{and} $\feasalts$ to winners $\emptyset \subset \alts \subseteq \feasalts$.
%	\end{description}
	\begin{itemize}
		\item Some alternatives might be infeasible
		\item Preferences over $\allalts$, choose among $\feasalts$
		\item $f$ a function that, given a profile and a set of feasible alternatives, chooses a subset of winners
		\item $f(\prof, \feasalts) \subseteq \feasalts$
	\end{itemize}
	\vfill
	\begin{center}
		\begin{tikzpicture}
			\path node (feas) {$\feasalts = \{a, c\}$};
			\path (feas.south) node[anchor=north, profile matrix] (profile) {
				\prof(1)&
				\prof(2)
				\\
				| (profile11) | a&
				b
				\\
				b&
				a
				\\
				c&
				| (profile32) | c
				\\
			};
			\path ($(profile.south west)!.5!(profile.south east)$) ++ (0, -5mm) node {$\prof$};
			
			\path node[draw, rectangle, fit=(profile11) (profile32), outer xsep=2mm, outer ysep=1mm] (justprofile) {};
			\path (justprofile.east) ++ (2.5cm, 0) node[inner sep=0] (winners) {\mbox{} $\alts = \set{a}$};
			\path[draw, ->] (justprofile.east) to[bend left=45] node[anchor=south] {$f$} (winners.west);
			\path[draw, ->] (feas.east) to[bend left=20] (winners.west);

			\path[draw, decorate, decoration={brace, mirror}] (justprofile.south west) -- (justprofile.south east);
		\end{tikzpicture}
	\end{center}
\end{frame}

\begin{frame}[fragile]
	\frametitle{\subsecname}
	\begin{definition}[\subsecname]
		When preferences change only about unfeasible alternatives, we ought not to change our decision
	\end{definition}
	\vspace{-1.8pt}
	\begin{example}
		\begin{equation}
			\begin{array}{l@{\hskip 0in}ll}
				\prof =
				&\begin{array}{rrr}
					b	&	c	&	d\\
					c	&	d	&	b\\
					a	&	b	&	a\\
					d	&	a	&	c
				\end{array},
				& f(\prof, \{b, c, d\}) = \set{b},\\[3em]
				\prof' =
				&\begin{array}{rrr}
					b	&	a	&	d\\
					c	&	c	&	a\\
					d	&	d	&	b\\
					a	&	b	&	c
				\end{array},
				& f(\prof', \{b, c, d\})\text{ ?}
				\pause
				\set{b}
			\end{array}
		\end{equation}
	\end{example}
\end{frame}

\subsection[ACA]{Arrow’s choice axiom}
\begin{frame}
	\frametitle{\subsecname}
	\begin{definition}[\subsecname]
		When feasible alternatives are restricted, with no change of preference, we ought not to change our decision if possible
	\end{definition}
	\vspace{-1.8pt}
	\begin{example}
		\begin{equation}
			\prof =
			\begin{array}{rrr}
				b	&	c	&	d\\
				c	&	d	&	b\\
				a	&	b	&	a\\
				d	&	a	&	c
			\end{array},
			f(\prof, \{b, c, d\}) = \set{b, d},
			f(\prof, \{b, c\})\text{ ?}
			\pause
			\set{b}
		\end{equation}
	\end{example}
\end{frame}

\subsection[Arrow]{Arrow’s impossibility theorem}
\begin{frame}
	\frametitle{Dictatorial social choice correspondences}
	\begin{definition}[Dictatorial social choice correspondence]
		$f$ is \emph{dictatorial} for a given voter iff $f$ always selects her preferred alternative among the feasible ones
	\end{definition}
	\vspace{-1.8pt}
	\begin{example}
		\begin{equation}
			\begin{array}{l@{\hskip 0in}ll}
				\prof =
				&\begin{array}{rrr}
					b	&	c	&	d\\
					c	&	d	&	b\\
					a	&	b	&	a\\
					d	&	a	&	c
				\end{array},
				& f(\prof, \{b, c, d\}) = \set{c},\\[3em]
				\prof' =
				&\begin{array}{rrr}
					b	&	a	&	d\\
					c	&	b	&	a\\
					d	&	d	&	b\\
					a	&	c	&	c
				\end{array},
				& f(\prof', \{b, c, d\})\text{ ?}
				\pause
				\set{b}
			\end{array}
		\end{equation}
	\end{example}
\end{frame}

\begin{frame}
	\frametitle{\subsecname}
	\begin{theorem}[\subsecname]
		When there are at least three alternatives, there is no social choice correspondence that satisfies Unanimity, Independence of Infeasible alternatives, Arrow’s Choice Axiom, and that is not dictatorial.
	\end{theorem}
\end{frame}

\section{Analyzing the Borda rule}
\subsection{Elementary profile}
\begin{frame}[fragile]
	\frametitle{\subsecname}
	
	Fix an arbitrary linear order $k$ on $\allalts$. Given $A \subseteq \allalts$, define $\pelem{A}$.
	\begin{minipage}{5.7cm}
		\begin{definition}[Elementary profile]
			\begin{tikzpicture}
				\path node[profile matrix, row 2/.style={nodes={draw, rectangle, minimum width=1.5em, minimum height=9ex}}, row 3/.style={nodes={draw, rectangle, minimum width=1.5em, minimum height=6ex}}] (profile) {
					\pelem{A}(1)	&\pelem{A}(2)\\
					| (profile-1-1) | 	& | (profile-1-2) | 	\\
					| (profile-2-1) | 	& | (profile-2-2) | 	\\
				};
				\path (profile-1-1.west) ++(-5mm, 0) node[anchor=east] (profile-1-1 expl) {$\restr{k}{A}$};
				\path (profile-1-2.east) ++(5mm, 0) node[anchor=west] (profile-1-2 expl) {$\restr{k^{-1}}{A}$};
				\path (profile-2-1.west) ++(-5mm, 0) node[anchor=east] (profile-2-1 expl) {$\restr{k}{\allalts \setminus A}$};
				\path (profile-2-2.east) ++(5mm, 0) node[anchor=west] (profile-2-2 expl) {$\restr{k^{-1}}{\allalts \setminus A}$};
				\path[draw, ->] (profile-1-1 expl.east) -- (profile-1-1.west);
				\path[draw, <-] (profile-1-2.east) -- (profile-1-2 expl.west);
				\path[draw, ->] (profile-2-1 expl.east) -- (profile-2-1.west);
				\path[draw, <-] (profile-2-2.east) -- (profile-2-2 expl.west);
				
				\path[draw, decorate, decoration={brace, mirror}] (profile.south west) -- (profile.south east);
				\path (profile.south) ++(0, -5mm) node {$\pelem{A}$};
			\end{tikzpicture}
		\end{definition}
	\end{minipage}\hspace{2em}%
	\begin{minipage}{\columnwidth - 5.7cm - 2em}
		\begin{example}
			\begin{tikzpicture}
				\path node[profile matrix] (profile) {
					\pelem{\Set{a, b, c}}(1)	&\pelem{\Set{a, b, c}}(2)\\
					| (profile-1-1) | a	& | (profile-1-2) | c	\\
					b	&b	\\
					| (profile-3-1) | c	& | (profile-3-2) | a	\\[2ex]
					| (profile-4-1) | d	& | (profile-4-2) | e	\\
					| (profile-5-1) | e	& | (profile-5-2) | d	\\
				};
				\path node[draw, rectangle, fit=(profile-1-1) (profile-3-1)] {};
				\path node[draw, rectangle, fit=(profile-4-1) (profile-5-1)] {};
				\path node[draw, rectangle, fit=(profile-1-2) (profile-3-2)] {};
				\path node[draw, rectangle, fit=(profile-4-2) (profile-5-2)] {};
				
				\path (profile.south west) ++ (1mm, -1mm) coordinate (profile-bottom-left);
				\path (profile.south east) ++ (-1mm, -1mm) coordinate (profile-bottom-right);
				\path[draw, decorate, decoration={brace, mirror}] (profile-bottom-left) -- (profile-bottom-right);
				\path (profile.south) ++(0, -5mm) node {$\pelem{\Set{a, b, c}}$};
			\end{tikzpicture}
		\end{example}
	\end{minipage}
\end{frame}

\begin{frame}
	\frametitle{Elementary profile: axiom}
	\begin{definition}[Axiom about elementary profiles]
		$f(\pelem{A}) = A$
	\end{definition}
\end{frame}

\subsection{Cyclic profile}
\begin{frame}
	\frametitle{\subsecname}
	\begin{definition}[Axiom about cyclic profiles]
		When given a cyclic profile, we ought to select all winners as ex-æquo
	\end{definition}
	\begin{example}
		\begin{equation}
			f\left(
			\begin{array}{cccc}
				a&b&c&d\\
				b&c&d&a\\
				c&d&a&b\\
				d&a&b&c\\
			\end{array}\right) = \allalts
		\end{equation}
	\end{example}
\end{frame}

\subsection{Cancellation}
\begin{frame}
	\frametitle{\subsecname}
	\begin{definition}[Cancellation]
		When all pairs of alternatives $(a, b)$ in a profile are such that $a$ is preferred to $b$ as many times as $b$ to $a$, we ought to select all winners as ex-æquo
	\end{definition}
	\begin{example}
		\begin{equation}
			f\left(%
			\begin{array}{rrrr}
				a&b&c&c\\
				b&a&a&b\\
				c&c&b&a\\
			\end{array}\right) = \allalts
		\end{equation}
	\end{example}
\end{frame}

\subsection{Reinforcement}
\begin{frame}
	\frametitle{\subsecname}
	\begin{definition}[\subsecname]
		When joining two sets of voters, exactly those winners that each set accepts should be selected, if possible
	\end{definition}
	\begin{example}
		\begin{equation}
			\prof_1 =
			\begin{array}{cc}
				a&b\\
				b&a\\
				c&c\\
			\end{array},
			A_1 = \set{a, b},
			\prof_2 =
			\begin{array}{ccc}
				a&b&a\\
				b&a&c\\
				c&c&b\\
			\end{array},
			A_2 = \set{a},
		\end{equation}
		\begin{equation}
			\prof =
			\begin{array}{ccccc}
				a&b&a&b&a\\
				b&a&b&a&c\\
				c&c&c&c&b\\
			\end{array}.
			\text{ Winners? }
			\pause
			\set{a}
		\end{equation}
	\end{example}
\end{frame}

\subsection{Characterisation of Borda}
\begin{frame}
	\frametitle{\subsecname}
	\begin{theorem}
		The Borda rule is the only voting rule that satisfies Elementary and Cyclic profile axioms, Cancellation and Reinforcement.
	\end{theorem}
	(We proved this variant of a characterisation given by \citet{young_axiomatization_1974})
\end{frame}

\subsection[Against Condorcet]{Fishburn-against-Condorcet argument}
\begin{frame}
	\frametitle{\subsecname}
	\begin{minipage}{5.4cm}
		\citet[p. 544]{fishburn_paradoxes_1974} argument against the Condorcet principle (see also \url{http://rangevoting.org/FishburnAntiC.html}).
		\begin{block}{Condorcet winner}
			$w \text{ VS } \mu, \mu \in \set{a, \ldots, h}\text{? }\uncover<2->{51/101}$
		\end{block}
	\end{minipage}\hfill%
	\mbox{%
		\small%
		$%
			\begin{array}{lrrrrrr}
				&\multicolumn{6}{c}{\text{nb voters}}\\
			\cmidrule{2-7}
					&31	&19	&10	&10	&10	&21	\\
			\midrule
				1	&a	&a	&f	&g	&h	&h	\\
				2	&b	&b	&w	&w	&w	&g	\\
				3	&c	&c	&a	&a	&a	&f	\\
				4	&d	&d	&h	&h	&f	&w	\\
				5	&e	&e	&g	&f	&g	&a	\\
				6	&w	&f	&e	&e	&e	&e	\\
				7	&g	&g	&d	&d	&d	&d	\\
				8	&h	&h	&c	&c	&c	&c	\\
				9	&f	&w	&b	&b	&b	&b	\\
			\end{array}
		$%
	}
	\centering
	$
		\begin{array}{lrrrrrrrrr}
			&\multicolumn{9}{c}{\text{ranks}}\\
		\cmidrule{2-10}
			&1	&≤2	&≤3	&≤4	&≤5	&≤6	&≤7	&≤8	&≤9	\\
		\midrule
		w	&0	& 30	& 30	& 51	& 51	& 82	& 82	& 82	&101	\\
		a	&50	& 50	& 80	& 80	& 101	& 101	& 101	&101	&101	\\
		\end{array}
	$
\end{frame}

\section{Our contribution}
\subsection{Goal}
\begin{frame}
	\frametitle{\subsecname}
	\begin{itemize}
		\item Better understanding of rules
		\item Illustrate reasoning of ≠ rules
		\item On examples
		\item On the basis of axioms
		\item Long-term goal: compare rules by letting them argue
	\end{itemize}
\end{frame}

\subsection{Approach}
\begin{frame}
	\frametitle{\subsecname}
	\begin{itemize}
		\item Express some axioms in a simple formal language
		\item Compute automatically their applications on examples
		\item Apply to the Borda rule
	\end{itemize}
\end{frame}

\subsection{Formal language}
\begin{frame}
	\frametitle{\subsecname}
	\begin{itemize}
		\item We use propositional logic
		\item A restricted language for easier deductions
		\item Contains $¬, ∨,∧, →$ but not $\forall, \exists$
		\item We translate axioms into this language
	\end{itemize}
\end{frame}

\subsection[Aim]{What we aim at}
\begin{frame}
	\frametitle{Example profile}
	\begin{block}{Who should win?}
		\begin{equation}
			\prof =
			\begin{array}{rrr}
				a&a&c\\
				b&b&b\\
				c&c&a\\
			\end{array}
		\end{equation}
	\end{block}
	\begin{itemize}
		\item Veto rule chooses $b$
		\item Borda rule chooses $a$
	\end{itemize}
\end{frame}

\begin{frame}
	\frametitle{\subsecname}
	\vspace{-13pt}
	\begin{equation}
		\prof =
		\begin{array}{>{\color{red}}r>{\color{darkgreen}}r>{\color{darkgreen}}r}
			a&a&c\\
			b&b&b\\
			c&c&a\\
		\end{array}
	\end{equation}
	\begin{tabularx}{\linewidth}{@{}>{\bfseries}lX>{\color{gray}[}r<{]}@{}}
		System: & Take the \textcolor{red}{\textit{red subprofile}}. Here, \textcolor{red}{\textit{$a$ should win}}, right? & unanimity \\
		User: & Obviously! \\
		System: & Now consider the \textcolor{darkgreen}{\textit{green subprofile}}. For symmetry reasons, there should be a \emph{three-way tie}, right? & cancellation \\
		User: & Sounds reasonable. \\
		System: &  So, as there was a three-way tie for the green part, the red part should decide the overall winner, right? & reinforcement \\
		User: & Yes. \\
		System: & \multicolumn{2}{l}{To summarise, you agree that $a$ should win.}
	\end{tabularx}
\end{frame}

\subsection{Provided}
\begin{frame}
	\frametitle{\subsecname}
	\begin{itemize}
		\item Our system can automatically explain the outcome of Borda
		\item Starting from any profile
		\item Requires solving a system of equation to find right intermediate profiles
		\item Does not use natural language (for the moment)
	\end{itemize}
\end{frame}

\subsection{Conclusion}
\begin{frame}
	\frametitle{\subsecname}
	\begin{itemize}
		\item No perfect voting rule
		\item Strengths can be expressed using axioms
		\item We use a specific language to express some of these axioms
		\item We can then instanciate concrete arguments (example-based)
		\item May render some arguments in the specialized literature accessible to non experts
		\item Extensions may permit to \emph{debate} about voting rules
		\item Provides a way to study appreciation of arguments
	\end{itemize}
\end{frame}

\begin{frame}[plain]
	\addtocounter{framenumber}{-1}
	\begin{center}
		\huge
		\textit{Thank you for your attention!}
	\end{center}
\end{frame}

\appendix
\AtBeginSection{
}

\begin{frame}[allowframebreaks]
	\frametitle{Bibliography}
	\def\newblock{\hskip .11em plus .33em minus .07em}
 	\bibliography{zotero}
\end{frame}

\section{Analyzing rules}
\subsection{Naïve attempt}
\begin{frame}
	\frametitle{\subsecname}
	⇒ Examine outcome of a rule on every possible profile?
	\begin{enumerate}
		\item We may not see what the problem is on a given example
		\item Too time consuming
	\end{enumerate}
	4 alternatives, 6 voters: at least $(4!)^6/4!/6!$ profiles $\eqsim \si{10000}$ {\tiny deducing obvious symetries for anonymity and neutrality}
% 4 voters, 4 alts: nb profiles = (4!)^4 but divided by approx 4! for permutations of alts, and divided by (very approx) 4! for permutations of voters, thus only 24×24 = 576.
\end{frame}

\section{Axioms}
\subsection{Anonymity}
\begin{frame}
	\frametitle{\subsecname}
	\begin{definition}[Anonymity]
		The identity of the voters should not matter
	\end{definition}
	\begin{equation}
		\prof_1 =
		\begin{array}{rrr}
			1&2&3\\
			a&b&b\\
			b&a&c\\
			c&c&a\\
		\end{array},
		\prof_2 =
		\begin{array}{ccc}
			3&1&2\\
			a&b&b\\
			b&a&c\\
			c&c&a\\
		\end{array}
	\end{equation}
	Constraint? \pause Same winners.
\end{frame}

\subsection{Cyclic profiles}
\begin{frame}
	\frametitle{\subsecname}
	Given $S$ a complete cycle in $\allalts$, define $\pcycl{S}$.
	\begin{definition}[Cyclic profile]
		$\pcycl{S}$ is the profile composed by all $\card{\allalts}$ possible linearizations of $S$ as preference orderings. 
	\end{definition}
	\begin{example}
		\begin{equation}
			\pcycl{\left<a, b, c, d\right>} =
			\begin{array}{cccc}
				a&b&c&d\\
				b&c&d&a\\
				c&d&a&b\\
				d&a&b&c\\
			\end{array}.
		\end{equation}
	\end{example}
\end{frame}

\subsection{Reinforcement}
\begin{frame}
	\frametitle{Reinforcement l-axiom}
	Consider $\prof_1$, $\prof_2$,
	\begin{itemize}
		\item having winners $A_1$, $A_2$,
		\item with $A_1 \cap A_2 ≠ \emptyset$;
	\end{itemize}
	then winners in $\prof_1 + \prof_2$ must be $A_1 \cap A_2$.
	\begin{definition}[\laxiom{Reinf}]
		For each $\prof_1, \prof_2, A_1, A_2 \subseteq \allalts, A_1 \cap A_2 ≠ \emptyset$:
		\begin{equation}
			(\lato[\prof_1][A_1] ∧ \lato[\prof_2][A_2]) → \lato[\prof_1 + \prof_2][A_1 \cap A_2].
		\end{equation}
	\end{definition}
\end{frame}

\section{Example using Borda}
\begin{frame}
	\frametitle{An example}
	
	Consider $\allalts=\set{a, b, c, d}$ and a profile $\prof$ defined as:
	\begin{equation}
		\prof =
		\begin{array}{cc}
			a	&	c\\
			b	&	b\\
			d	&	a\\
			c	&	d
		\end{array}.
	\end{equation}
	\pause
	We want to justify that $f_B(\prof)=\Set{a, b}$.
\end{frame}

\begin{frame}
	\frametitle{Sketch}
	
	\begin{itemize}
		\item Consider any $\prof' = q_1 \pelem{a, b} + q_2 \pelem{a, b, c} + \sum_{S \in \mathcal{S}} q_S \pcycl{S}$, $q_1, q_2, q_S \in \N, \mathcal{S}$ some set of cycles.
		\item In $\prof'$, $W=\set{a, b}$ must win.
		\item Find $k \in \N$ such that $\overline{k \prof} + \prof'$ cancel.
		\item Then $k \prof$ has winners $W$. (Skipping details.)
		\item Then $\prof$ has winners $W$.
	\end{itemize}
	Our task: find $\prof'$ a combination of elementary and cyclic profiles such that $\overline{k \prof} + \prof'$ cancel.
	
	Good news: this is always possible.
\end{frame}

\begin{frame}
	\frametitle{Application on the example}
	
	Define $\prof' = \pelem{a, b} + 2 \pelem{a, b, c} + \pcycl{\left<c, b, a, d\right>} + \pcycl{\left<b, d, c, a\right>}$.
	\begin{enumerate}
		\item $\lato[\pelem{a, b}][\set{a, b}]$ (\laxiom{elem})
		\item $\lato[\pelem{a, b, c}][\set{a, b, c}]$ (\laxiom{elem})
		\item $\lato[\pcycl{\left<c, b, a, d\right>}][\allalts]$ (\laxiom{cycl})
		\item $\lato[\pcycl{\left<b, d, c, a\right>}][\allalts]$ (\laxiom{cycl})
		\item $\lato[\prof'][\set{a, b}]$ (\laxiom{reinf}, 1, 2, 3, 4)
		\item $\lato[4 \prof + \overline{4 \prof}][\allalts]$ (\laxiom{canc})
		\item $\lato[4 \prof + \overline{4 \prof} + \prof'][\set{a, b}]$ (\laxiom{reinf}, 5, 6)
		\item $\lato[\overline{4 \prof} + \prof'][\allalts]$ (\laxiom{canc})
		\item $\lato[4 \prof][\set{a, b}]$ (\laxiom{reinf}, 7, 8)
		\item $\lato[\prof][\set{a, b}]$ (\laxiom{reinf}, 9)
	\end{enumerate}
\end{frame}

\section{License}
\begin{frame}
	\frametitle{License}
	This presentation, and the associated \LaTeX{} code, are published under the \href{http://opensource.org/licenses/MIT}{MIT license}. Feel free to reuse (parts of) the presentation, under condition that you cite the author.
	
	Credits are to be given to \href{http://www.lamsade.dauphine.fr/~ocailloux/}{Olivier Cailloux}, Université Paris-Dauphine.
\end{frame}
\end{document}

\section{Frame template}
\subsection{}
\begin{frame}
	\frametitle{\subsecname}
	\begin{itemize}
		\item 
	\end{itemize}
\end{frame}

